\documentclass{article}
\usepackage{amsmath}
\usepackage{geometry}
\geometry{a4paper, margin=1in}
\usepackage{siunitx} % For units

\begin{document}

\section*{HP-1 Engine Preliminary Analysis Report: Parts B \& C}

This report details the analysis and corrected calculations for Task B (Maximum Allowable TSFC) and Task C (Minimum Specific Thrust) based on the requirements outlined in the HP-1 design problem manual \cite{manual}.

\section{Task B: Maximum Allowable TSFC}

The maximum allowable Thrust Specific Fuel Consumption (TSFC) for the cruise segment is determined using the Breguet range equation. Key parameters and corrections based on the manual are:
\begin{itemize}
    \item Cruise Mach Number: $M = 0.83$
    \item Cruise Altitude: $h = \SI{11}{km}$ ($T = \SI{216.65}{K}$, $a = \SI{295.07}{m/s}$)
    \item Cruise Speed: $V = M \times a = 0.83 \times \SI{295.07}{m/s} = \SI{244.91}{m/s}$
    \item Cruise Segment Range: $R = \SI{10650}{km}$ (from Table 1 \cite{manual})
    \item Start of Cruise Weight: $W_{\text{start}} = \SI{1577940}{N}$ (approx. \SI{160850}{kg}) \cite{manual}
    \item Cruise Fuel: $\SI{50240}{kg}$ (from Table 1 \cite{manual})
    \item End of Cruise Weight: $W_{\text{end}} = W_{\text{start}} - (\text{Cruise Fuel} \times g) = \SI{160850}{kg} - \SI{50240}{kg} = \SI{110610}{kg}$
    \item Cruise Weight Fraction: $W_{\text{end}} / W_{\text{start}} = 110610 / 160850 = 0.6877$
    \item Assumed Lift-to-Drag Ratio: $L/D = 17.0$ (User assumption, requires verification)
    \item Installation Loss Factor: $\Phi_{\text{loss}} = 0.02$ \cite{manual}
\end{itemize}

The Breguet range equation is:
\[ R = \frac{V}{S_{\text{installed}}} \left( \frac{L}{D} \right) \ln \left( \frac{W_{\text{start}}}{W_{\text{end}}} \right) \]
Rearranging for installed TSFC ($S_{\text{installed}}$ in \SI{}{kg/(N.s)}):
\[ S_{\text{installed}} = \frac{V}{R} \left( \frac{L}{D} \right) \ln \left( \frac{W_{\text{start}}}{W_{\text{end}}} \right) \]
\[ S_{\text{installed}} = \frac{\SI{244.91}{m/s}}{\SI{10650e3}{m}} (17.0) \ln \left( \frac{1}{0.6877} \right) \]
\[ S_{\text{installed}} = (\SI{2.2996e-5}{s^{-1}}) (17.0) (0.3744) = \SI{1.4620e-4}{kg/(N.s)} \]

The uninstalled TSFC ($S$) is:
\[ S = \frac{S_{\text{installed}}}{1 - \Phi_{\text{loss}}} = \frac{\SI{1.4620e-4}{kg/(N.s)}}{1 - 0.02} = \frac{\SI{1.4620e-4}{kg/(N.s)}}{0.98} \]
\[ S = \SI{1.4918e-4}{kg/(N.s)} = \SI{14.92}{mg/(N.s)} \]

\textbf{Result Task B:} The maximum allowable uninstalled TSFC is \textbf{\SI{14.92}{mg/(N.s)}}. This value is realistic for modern high-bypass turbofan engines at cruise conditions. The original calculation result (\SI{17.08}{mg/(N.s)}) differed primarily due to using incorrect initial assumptions for cruise weight fraction and range.

\section{Task C: Minimum Specific Thrust}

The minimum required specific thrust ($F/ \dot{m}_{\text{air}}$) is determined by the required engine thrust at the start of cruise and the maximum possible air mass flow through the engine inlet.

\subsection{Required Uninstalled Thrust per Engine}
\begin{itemize}
    \item Start of Cruise Weight: $W = \SI{1577940}{N}$ \cite{manual}
    \item Lift-to-Drag Ratio: $L/D = 17.0$ (Assumed)
    \item Drag: $D = W / (L/D) = \SI{1577940}{N} / 17.0 = \SI{92820}{N}$
    \item Required Rate of Climb: $P_s = \SI{1.5}{m/s}$ \cite{manual}
    \item Cruise Speed: $V = \SI{244.91}{m/s}$
    \item Required Installed Thrust (Total): $T_{\text{installed}} = D + \frac{W \times P_s}{V} = \SI{92820}{N} + \frac{\SI{1577940}{N} \times \SI{1.5}{m/s}}{\SI{244.91}{m/s}} = \SI{92820}{N} + \SI{9665}{N} = \SI{102485}{N}$
    \item Required Uninstalled Thrust (Total): $T_{\text{uninstalled, total}} = \frac{T_{\text{installed}}}{1 - \Phi_{\text{loss}}} = \frac{\SI{102485}{N}}{1 - 0.02} = \SI{104577}{N}$
    \item Required Uninstalled Thrust per Engine (2 engines): $T_{\text{uninstalled, eng}} = T_{\text{uninstalled, total}} / 2 = \SI{104577}{N} / 2 = \SI{52288}{N}$
\end{itemize}
This required thrust (\SI{52288}{N}/engine) is higher than the originally calculated \SI{48744}{N}/engine due to the use of the correct start-of-cruise weight.

\subsection{Maximum Inlet Mass Flow}
The maximum mass flow per unit area is given by the manual \cite{manual}:
\[ \frac{\dot{m}}{A} = 231.8 \frac{\delta_{0}}{\sqrt{\theta_{0}}} \quad \text{(\si{kg/s}/m^2)} \]
Where $\theta_0 = T_{t0}/T_{SL}$ and $\delta_0 = P_{t0}/P_{SL}$. At $M=0.83$, $h=\SI{11}{km}$ ($T=\SI{216.65}{K}$, $P=\SI{22631.7}{Pa}$):
\begin{itemize}
    \item $T_{t0} = T (1 + \frac{\gamma-1}{2} M^2) = 216.65 (1 + 0.2 \times 0.83^2) = \SI{246.49}{K}$
    \item $P_{t0} = P (1 + \frac{\gamma-1}{2} M^2)^{\gamma/(\gamma-1)} = 22631.7 (1 + 0.2 \times 0.83^2)^{3.5} = \SI{35611}{Pa}$
    \item $\theta_0 = T_{t0} / \SI{288.15}{K} = 246.49 / 288.15 = 0.8554$
    \item $\delta_0 = P_{t0} / \SI{101325}{Pa} = 35611 / 101325 = 0.3514$
\end{itemize}
\[ \frac{\dot{m}}{A} = 231.8 \frac{0.3514}{\sqrt{0.8554}} = 231.8 \times \frac{0.3514}{0.9249} = \SI{88.01}{kg/s/m^2} \]

\subsection{Minimum Specific Thrust Calculation}
The minimum uninstalled specific thrust is $F_{\text{uninstalled}} / \dot{m}_{\text{air, max}} = T_{\text{uninstalled, eng}} / (A_{\text{inlet}} \times (\dot{m}/A)_{\text{max}})$.

The manual states a maximum allowable inlet diameter of \SI{2.2}{m} \cite{manual}. Calculations for larger diameters are included for comparison but violate this constraint.

\begin{table}[h!]
\centering
\caption{Recalculated Minimum Specific Thrust vs. Inlet Diameter}
\begin{tabular}{|c|c|S[table-format=3.1]|S[table-format=3.1]|c|}
\hline
Diameter (\si{m}) & Inlet Area (\si{m^2}) & {Max Mass Flow (\si{kg/s})} & {Min Spec. Thrust (\si{N.s/kg})} & Feasible \\
\hline
\textbf{2.2} & 3.801 & 334.5 & \textbf{156.3} & \textbf{Yes} \\
2.5 & 4.909 & 432.0 & 121.0 & No \\
2.75 & 5.940 & 522.8 & 99.0 & No \\
3.0 & 7.069 & 622.1 & 84.1 & No \\
3.25 & 8.296 & 730.1 & 71.6 & No \\
3.5 & 9.621 & 846.7 & 61.7 & No \\
\hline
\end{tabular}
\label{tab:spec_thrust}
\end{table}

\textbf{Result Task C:} Considering the constraint $D_{\text{inlet}} \le \SI{2.2}{m}$, the minimum allowable uninstalled specific thrust at the start of cruise is \textbf{\SI{156.3}{N.s/kg}}. The specific thrust values decrease with increasing inlet diameter (mass flow) for the fixed thrust requirement. The recalculated values differ from the original calculations due to the corrected thrust requirement and the application of the specific mass flow formula from the manual.

\begin{thebibliography}{9}
    \bibitem{manual} Provided `Gas Turbine Design Problem HP-1` file detailing the HP-1 Aircraft Design Problem.
    \bibitem{book} Elements of Propulsion, Gas Turbines and Rockets 2nd edition.
\end{thebibliography}

\end{document}
